%%%%%%%%%%%%%%%%%%%%%%%%%%%%%%%%%%%%%%%%%
% Journal Article
% LaTeX Template
% Version 1.2 (15/5/13)
%
% This template has been downloaded from:
% http://www.LaTeXTemplates.com
%
% Original author:
% Frits Wenneker (http://www.howtotex.com)
%
% License:
% CC BY-NC-SA 3.0 (http://creativecommons.org/licenses/by-nc-sa/3.0/)
%
%%%%%%%%%%%%%%%%%%%%%%%%%%%%%%%%%%%%%%%%%

%----------------------------------------------------------------------------------------
%	PACKAGES AND OTHER DOCUMENT CONFIGURATIONS
%----------------------------------------------------------------------------------------

\documentclass[twoside]{article}

\usepackage{graphicx}
\usepackage{xcolor}
\usepackage{listings}

\definecolor{grey}{rgb}{0.9,0.9,0.9}
\lstset{basicstyle=\ttfamily,
  showstringspaces=false,
  backgroundcolor=\color{grey},
  commentstyle=\color{red},
  keywordstyle=\color{blue}
}

%\usepackage{lipsum} % Package to generate dummy text throughout this template

\usepackage[sc]{mathpazo} % Use the Palatino font
\usepackage[T1]{fontenc} % Use 8-bit encoding that has 256 glyphs
\linespread{1.05} % Line spacing - Palatino needs more space between lines
\usepackage{microtype} % Slightly tweak font spacing for aesthetics

\usepackage[hmarginratio=1:1,top=32mm,columnsep=20pt]{geometry} % Document margins
\usepackage{multicol} % Used for the two-column layout of the document
\usepackage[hang, small,labelfont=bf,up,textfont=it,up]{caption} % Custom captions under/above floats in tables or figures
\usepackage{booktabs} % Horizontal rules in tables
\usepackage{float} % Required for tables and figures in the multi-column environment - they need to be placed in specific locations with the [H] (e.g. \begin{table}[H])
\usepackage{hyperref} % For hyperlinks in the PDF

\usepackage{lettrine} % The lettrine is the first enlarged letter at the beginning of the text
\usepackage{paralist} % Used for the compactitem environment which makes bullet points with less space between them

\usepackage{abstract} % Allows abstract customization
\renewcommand{\abstractnamefont}{\normalfont\bfseries} % Set the "Abstract" text to bold
\renewcommand{\abstracttextfont}{\normalfont\small\itshape} % Set the abstract itself to small italic text

\usepackage{titlesec} % Allows customization of titles
\renewcommand\thesection{\Roman{section}}
\titleformat{\section}[block]{\large\scshape\centering}{\thesection.}{1em}{} % Change the look of the section titles

\usepackage{fancyhdr} % Headers and footers
\pagestyle{fancy} % All pages have headers and footers
\fancyhead{} % Blank out the default header
\fancyfoot{} % Blank out the default footer
\fancyhead[C]{Flex Board Meets OBD $\bullet$ August 2013} % Custom header text
\fancyfoot[RO,LE]{\thepage} % Custom footer text

%----------------------------------------------------------------------------------------
%	TITLE SECTION
%----------------------------------------------------------------------------------------

\title{\vspace{-15mm}\fontsize{24pt}{10pt}\selectfont\textbf{Flex Board Meets OBD}} % Article title

\author{
\large
\textsc{Alessio Balsini}\\[2mm] % Your name
\textsc{David Librera}\\[2mm] % Your name
\normalsize Universit\`a degli Studi di Pisa\\ % Your institution
\normalsize Scuola Superiore Sant'Anna\\ % Your institution
\normalsize \href{mailto:a.balsini@sssup.it}{a.balsini@sssup.it} % Your email address
\normalsize \href{mailto:d.librera@sssup.it}{d.librera@sssup.it} % Your email address
\vspace{-5mm}
}
\date{}

%----------------------------------------------------------------------------------------

\begin{document}

\maketitle % Insert title

\thispagestyle{fancy} % All pages have headers and footers

%----------------------------------------------------------------------------------------
%	ABSTRACT
%----------------------------------------------------------------------------------------

\begin{abstract}

\noindent The aim of the project is the development of a system which retrieves and presents to the user diagnostic data from a vehicle. This system is made of Flex Board developed by Evidence, Bluetooth-to-UART module and a Bluetooth Elm327 device. Information gathered are presented to the user by an LCD display embedded in the Flex Board or by an interactive graphical interface on PC.

\end{abstract}

%----------------------------------------------------------------------------------------
%	ARTICLE CONTENTS
%----------------------------------------------------------------------------------------

\begin{multicols}{2} % Two-column layout throughout the main article text

\section{Introduction}

\lettrine[nindent=0em,lines=2]{N} owadays almost all the vehicles adopt, by law, a standard way of communicating diagnostic information.

The OBDII standard is actually the most common protocol. It allows the user not only to retrieve diagnostic data, but also to program the control units and modify vehicle's parameters in real time.

%------------------------------------------------

\section{Hardware}

\subsection{Modules Involved}

Many interfaces for transfering vehicle diagnostics are not directly available in common PCs or PDAs.

As bridge between OBDII and Bluetooth it is possible to use Elm 327 modules.

\begin{figure}[H]
  \centering
  \includegraphics[width=2in]{img/elm_327_presentation}
  \caption{\textit{Elm 327} Bluetooth/OBDII module}
\end{figure}

By default, our module is discovered with the Bluetooth name "CHX" and requires the pin code "6789" for pairing.

The Flex Light is the core of our bridge.

It comes with a dsPIC by Microchip and is programmed with MPLAB ICD tools.

\begin{figure}[H]
  \centering
  \includegraphics[width=2.6in]{img/flex_light_presentation}
  \caption{\textit{Flex Light} board}
\end{figure}

Flex Light functionalities can be extended with the Multibus board which provides different useful communication interfaces.

In particular, in this project will be used the two UART ports.

\begin{figure}[H]
  \centering
  \includegraphics[width=2.6in]{img/flex101_presentation}
  \caption{\textit{Flex Multibus} daughter board}
\end{figure}

Another daughter board which extends Flex Light's functionalities is the Flex Demo Board.

\begin{figure}[H]
  \centering
  \includegraphics[width=2.3in]{img/flex_demo_board_presentation}
  \caption{\textit{Flex Demo} board}
\end{figure}

The Bluetooth communication is allowed by the BlueSmirf module, a little board which embeds a Roving Networks RN-42.

\begin{figure}[H]
  \centering
  \includegraphics[width=2.3in]{img/bluesmirf_presentation}
  \caption{\textit{BlueSmirf} Bluetooth module}
\end{figure}

\subsection{Connectors and Wirings}

The Flex Demo Board pins involved in the project are the following:

\begin{figure}[H]
  \centering
  \includegraphics[width=2in]{img/flex_demo_board_pinout}
  \caption{\textit{Flex Demo Board} pinout}
\end{figure}

\begin{compactitem}
\item pin 1: 5 V;
\item pin 2: TX-O, UART output;
\item pin 3: SCK, Serial ClocK;
\item pin 4: RX-I, UART input;
\item pin 5: i.e. Vout;
\item pin 6: i.e. GND.
\end{compactitem}

The Bluetooth device uses has the following pinout:

\begin{figure}[H]
  \centering
  \includegraphics[width=3in]{img/bluesmirf_pinout}
  \caption{\textit{Bluesmirf} (Bluetooth/UART module) pinout}
\end{figure}

\begin{compactitem}
\item pin 1: CTS-I, Clear to Send (handshake pin);
\item pin 2: VCC (from 3.3V to 6V);
\item pin 3: GND;
\item pin 4: TX-O, UART output;
\item pin 5: RX-I, UART input;
\item pin 6: RTS-O, Request to Send (handshake pin).
\end{compactitem}

The Bluesmirf module has been connected to the Flex Demo Board with the following wiring:

\begin{compactitem}
\item Flex RX-I <-> Bt TX-O;
\item Flex TX-O <-> Bt RX-I;
\item Flex 5 V <-> Bt Vcc;
\item Flex GND <-> Bt GND.
\end{compactitem}

The standard OBDII female connector has the following pinout:

\begin{figure}[H]
  \centering
  \includegraphics[width=2.5in]{img/J1962_female_pinout}
  \caption{OBD female connector pinout}
\end{figure}

\begin{compactitem}
\item pin 1: NC;
\item pin 2: Bus+ (J1850);
\item pin 3: NC;
\item pin 4: GND (chassis);
\item pin 5: GND (signal);
\item pin 6: CAN High (J-2284);
\item pin 7: K-Line (ISO9141-2);
\item pin 8: NC;
\item pin 9: NC;
\item pin 10: Bus (J1850);
\item pin 11: NC;
\item pin 12: NC;
\item pin 13: NC;
\item pin 14: CAN Low (J-2284);
\item pin 15: L-Line (ISO9141-2);
\item pin 16: Board Power (12 V).
\end{compactitem}


%------------------------------------------------

\section{Background}

\subsection{Elm327 Communication Basics}

Elm327 accepts basically two types of input:
\begin{compactitem}
  \item commands: allow various kind of operations on device;
  \item data requests: allow to retrieve data from the vehicle.
\end{compactitem}

Elm327 is case-insensitive and does not care about spacing symbols (tabs or white spaces).
Nevertheless, in this document will be used uppercases and a correct amount of spacing, with the aim of improving readability.

\subsubsection{Basic Commands}

All commands are prefixed with "AT". It is possible to distinguish between four classes of commands:
\begin{compactitem}
  \item general commands;
  \item OBD commands;
  \item protocols specific commands;
  \item miscellaneous commands.
\end{compactitem}

The syntax for commands is, usally
\begin{lstlisting}[language=bash]
AT CMD [param]
\end{lstlisting}

\begin{lstlisting}[language=bash]
AT Z
\end{lstlisting}

Resets the module, as if a power reset happened.
It brings back all the settings to their default values.
At reboot, module prints out the device name and firmware version, like \emph{ELM327 v1.4}.

\begin{lstlisting}[language=bash]
AT SP x
\end{lstlisting}

Sets the protocol to \emph{x}, in particular, by default:
\begin{compactitem}
  \item 0 automatic;
  \item 1 SAE J1850 PWM (41.6 Kbaud);
  \item 2 SAE J1850 VPW (10.4 Kbaud);
  \item 3 ISO 9141-2 (5 baud init, 10.4 Kbaud);
  \item 4 ISO 14230-4 KWP (5 baud init, 10.4 Kbaud);
  \item 5 ISO 14230-4 KWP (fast init, 10.4 Kbaud);
  \item 6 ISO 15765-4 CAN (11 bit ID, 500 Kbaud);
  \item 7 ISO 15765-4 CAN (29 bit ID, 500 Kbaud);
  \item 8 ISO 15765-4 CAN (11 bit ID, 250 Kbaud);
  \item 9 ISO 15765-4 CAN (29 bit ID, 250 Kbaud);
  \item A SAE J1939 CAN (29 bit ID, 250 Kbaud);
  \item B USER1 CAN (11 bit ID, 125 Kbaud);
  \item C USER2 CAN (11 bit ID, 50 Kbaud).
\end{compactitem}

\subsubsection{Basic Data}

Each data request has three parameters associated:
\begin{compactitem}
  \item mode: hexadecimal 2 byte identifier;
  \item pid: hexadecimal 2 byte identifier;
  \item response length: response bytes.
\end{compactitem}

Each request has the following syntax
\begin{lstlisting}[language=bash]
MODE PID
\end{lstlisting}
and each response has the syntax
\begin{lstlisting}[language=bash]
MODE PID VALUE
\end{lstlisting}
where \emph{VALUE} has different fixed sizes for different data.
\emph{VALUE}'s length is \emph{n}-byte, each byte separated by a space symbol.

\begin{lstlisting}[language=bash]
00 PID_REQ
\end{lstlisting}

where
\begin{equation}
PID\_REQ = 0x10 \cdot k , \quad k = 0,1,2,4
\end{equation}
Requests the list of pids available.

The value returned is a bit map, where each bit is associated to a pid and has value:
\begin{compactitem}
  \item 1: available;
  \item 0: not available.
\end{compactitem}


\begin{lstlisting}[language=bash]
01 01
\end{lstlisting}
Requests the list of error codes (\emph{DTCs}, \emph{Diagnostic Trouble Codes}).
This returns a 4 byte response, like
\begin{lstlisting}[language=bash]
41 01 AA BB CC DD
\end{lstlisting}
where:
\begin{compactitem}
  \item \emph{41 01}: is the response to the request;
  \item \emph{AA}: is the number of trouble codes:
  \begin{compactitem}
    \item most significant bit: Multifunction Indicator Lamp (MIL, aka Check Engine Light) has been turned on;
    \item remaining bits: number of stored trouble codes, flagged in the ECU (Engine Control Unit);
  \end{compactitem}
  \item \emph{BB CC DD}: information about the availability and completeness of certain on-board tests, with different interpretations in case of spark or compression ignition engines. More details on SAE J1979 documentation.
\end{compactitem}

So, with the \emph{AA} value set to \emph{81} hex (\emph{10000001} bin) it is possible to recognize that there is one trouble code stored which turned on the Check Engine Light.

\begin{table}[H]
\centering
\begin{tabular}{llr}
\multicolumn{3}{c}{BB} \\
\toprule
Test & Available & Incomplete \\
\midrule
Misfire & B0 & B4 \\          
Fuel System & B1 & B5 \\
Components & B2 & B6 \\
\bottomrule
\end{tabular}
\end{table}

\begin{table}[H]
\centering
\begin{tabular}{llr}
\multicolumn{3}{c}{CC, DD in Spark Ignition} \\
\toprule
Test & Avail. & Incompl. \\
\midrule
Catalyst & C0 & D0 \\
Heated Catalyst & C1 & D1 \\
Evaporative System & C2 & D2 \\
Secondary Air System & C3 & D3 \\
A/C Refrigerant & C4 & D4 \\
Oxygen Sensor & C5 & D5 \\
Oxygen Sensor Heater & C6 & D6 \\
EGR System & C7 & D7 \\
\bottomrule
\end{tabular}
\end{table}

\begin{table}[H]
\centering
\begin{tabular}{llr}
\multicolumn{3}{c}{CC, DD in Compression Ignition} \\
\toprule
Test & Avail. & Incompl. \\
\midrule
NMHC Cat & C0 & D0 \\
NOx/SCR Monitor & C1 & D1 \\
Boost Pressure & C3 & D3 \\
Exhaust Gas Sensor & C5 & D5 \\
PM filter monitoring & C6 & D6 \\
EGR and/or VVT System & C7 & D7 \\
\bottomrule
\end{tabular}
\end{table}

\section{FlexMeetsObd}

\subsection{PC Interface}

\subsubsection{Overview}

When the PC software is launched, the user will obtain the welcome window shown in figure.

\begin{figure}[H]
  \centering
  \includegraphics[width=3in]{img/GUI/welcome_window}
  \caption{Welcome window}
\end{figure}

Monitors are sub-windows that will be drawn inside the dark grey region of the main window. Each monitor is used to represent the vehicle data associated to it and some additional information.

From each monitor is possible to open a new window containing it's plot diagram, which shows the last received samples.

For monitors management, click on \emph{View > Select Monitors} or directly on \emph{Select Monitors} on the right bar menu.

\begin{figure}[H]
  \centering
  \includegraphics[width=2.2in]{img/GUI/monitor_selection}
  \caption{Monitor selection window}
\end{figure}

Once the monitors are chosen, the main window will be populated.

\begin{figure}[H]
  \centering
  \includegraphics[width=3in]{img/GUI/monitor_created}
  \caption{Mainwindow after monitor creation}
\end{figure}

At this point, monitor has inconsistent values.

It is possible to move monitors inside the window. To make an automatic monitors alignment, click on \emph{View > Tile Monitors} or on \emph{Tile Monitors} on the left bar.

\subsubsection{Connections}

It's time to estabilish the connections.

The first step is to configure the serial port, through \emph{Configure > Serial Configuration}.

Remember to set appropriate permissions to the file descriptor associated to the serial port in \emph{/dev/} folder.

\begin{figure}[H]
  \centering
  \includegraphics[width=3in]{img/GUI/serial_port_configuration}
  \caption{Serial port configuration window}
\end{figure}


The coloured dots in the status bar, in bottom right of the window, show the status of the Serial, Flex and Vehicle links.
Pay attention to them, in order to detect if something is going wrong.

To enable [disable] serial device, click on \emph{Devices > Serial Enable [Disable]} or on \emph{Serial Enable [Disable]} on the left bar.
The Serial dot should become green, otherwise, check if the serial device is properly connected and permissions are correct.

To establish the connection with the Flex, click on \emph{Devices > Bridge Connect} or on \emph{Bridge Connect} on the left bar.
The Flex dot should become green, otherwise, check if the serial configuration parameters are compatible with Flex's ones, Flex is powered on, Flex is in a consistent state and Flex is correctly connected to PC.

Not it is possible to request the Flex to perform a Bluetooth inquiry, searching for the Elm327 device.

Inquiry results are presented to the user, after clicking on \emph{Devices > Bridge Inquiry} or on \emph{Bridge Inquiry} on the left bar.

\begin{figure}[H]
  \centering
  \includegraphics[width=2.8in]{img/GUI/bluetooth_scan_results}
  \caption{Bluetooth scan results}
\end{figure}

Select the preferred device in list and click \emph{Ok}.

If the device is not in list, click \emph{Cancel}, check if Flex and Elm327 are correctly working and perform another inquiry.

\subsubsection{Samples Presentation}

Once the Elm327 device is chosen inside the inquiry list, data Flex will periodically send data to PC.

This data will be shown to the user similarly to how is shown in figures.

\begin{figure}[H]
  \centering
  \includegraphics[width=3in]{img/GUI/sampling_monitor}
  \caption{Monitor appearance while sampling}
\end{figure}

\begin{figure}[H]
  \centering
  \includegraphics[width=3in]{img/GUI/sampling_plot}
  \caption{Plot appearance while sampling}
\end{figure}

It is always possible to add new monitors or remove monitors from main window, also while sampling.

\subsection{Flex Multibus}

\subsection{Flex Demo Board Standalone}

After powerup, user receives a welcome message.

\begin{lstlisting}[language=bash]
  Flex2OBD 0.1
    WELCOME!
\end{lstlisting}

Press any key to continue.

Now Flex is going to:
\begin{compactitem}
  \item initialize the UART port;
  \item initialize the RN-42 module;
  \item setup the Bluetooth connection parameters;
  \item perform an inquiry.
\end{compactitem}

\begin{lstlisting}[language=bash]
BT Init...[Done]
SetMode...[Done]
Inquiry...[Done]
\end{lstlisting}

If no device is found, push any key to rescan.

Otherwise, inquiry results will be shown to the user with an interactive menu. 

\begin{lstlisting}[language=bash]
Devices found: 3
@ Elm327
\end{lstlisting}

It is possible to navigate the menu using the buttons, with a vi-like meaning:
\begin{compactitem}
  \item button 0: left;
  \item button 1: down;
  \item button 2: up;
  \item button 3: right.
\end{compactitem}

Going up or down shows the devices found.

Pressing left or right navigates from the module's properties: from left to right, name, address and class.

Before the name is shown an \emph{@} symbol, meaning that pressing left will estabilish a connection with that device.
After the class is shown a \emph{!} symbol, meaning that pressing the right key will perform another inquiry.

After connection request, another menu will appear.

\begin{lstlisting}[language=bash]
Estabilish Conn.
Comm|RConn|RScan
\end{lstlisting}

By pressing second or third button will attempt another connection with the device.

By pressing last button will perform another inquiry.

By pressing first button will start the communication with the Elm327, sentting up the module and requesting periodically the vehicle data. Before pressing, check if the Bluesmirf's led is green, otherwise something went wrong during the connection.

If connection is estabilished, the welcome string of the Elm will be shown.

\begin{lstlisting}[language=bash]
Elm v.1.4
\end{lstlisting}

%\begin{table}[H]
%\caption{Example table}
%\centering
%\begin{tabular}{llr}
%\toprule
%\multicolumn{2}{c}{Name} \\
%\cmidrule(r){1-2}
%First name & Last Name & Grade \\
%\midrule
%John & Doe & $7.5$ \\
%Richard & Miles & $2$ \\
%\bottomrule
%\end{tabular}
%\end{table}

%\lipsum[5] % Dummy text

%\begin{equation}
%\label{eq:emc}
%e = mc^2
%\end{equation}

%\lipsum[6] % Dummy text

%----------------------------------------------------------------------------------------
%	REFERENCE LIST
%----------------------------------------------------------------------------------------

\begin{thebibliography}{99} % Bibliography - this is intentionally simple in this template

\bibitem{} A. S. Huang, L. Rudolph,
  \emph{Bluetooth Essentials for Programmers},
  Cambridge University Press (2007)

\bibitem{}
  \emph{Bluetooth Data Module Command Reference \& Advanced Information User's Guide},
  Roving Networks (2013)
  
\bibitem{}
  \emph{ELM327 - OBD to RS232 Interpreter},
  Elm Electronics
 
\end{thebibliography}

%----------------------------------------------------------------------------------------

\end{multicols}

\end{document}
